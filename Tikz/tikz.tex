% function to compute angle and length between coordinates
\makeatletter
\newcommand{\getLengthAndAngle}[2]{%
    \pgfmathanglebetweenpoints{\pgfpointanchor{#1}{center}}
                              {\pgfpointanchor{#2}{center}}
    \global\let\myangle\pgfmathresult % we need a global macro
    \pgfpointdiff{\pgfpointanchor{#1}{center}}
                 {\pgfpointanchor{#2}{center}}
    \pgf@xa=\pgf@x % no need to use a new dimen
    \pgf@ya=\pgf@y
    \pgfmathparse{veclen(\pgf@xa,\pgf@ya)/28.45274} % to convert from pt to cm
    \global\let\mylength\pgfmathresult % we need a global macro
}
\makeatother 
\tikzset{
    pics/pivot/.style args={#1,#2,#3}{
		code = {
			% #1: radius of small circle (pin);
			% #2: radius of large circle (outer dimension).
            % #3: rotation angle
			\draw[thick, fill=white, rotate=#3] (#2,-#2) -- (#2,0) arc(0:180:#2) -- (-#2,-#2) -- cycle;
			\draw[thick, fill=gray] (0,0) circle circle (#1);
		}
	},
	ground/.style = {fill,pattern=north east lines,draw=none,minimum width=1cm,minimum height=1cm},
    spring/.style = {thick,decorate,decoration={zigzag,pre length=4pt,post length=4pt,segment length=3pt}},
    damper/.style = {thick, decoration={markings, mark connection node=dmp, mark=at position 0.5 with
              {
                \node (dmp) [thick,inner sep=0pt,transform shape,rotate=-90,minimum width=8pt,minimum height=2pt,draw=none] {};
                \draw [thick] ($(dmp.north east)+(2pt,0)$) -- (dmp.south east) -- (dmp.south west) -- ($(dmp.north west)+(2pt,0)$);
                \draw [thick] ($(dmp.north)+(0,-3pt)$) -- ($(dmp.north)+(0,3pt)$);
              }
        }, decorate},
	pics/dimLinear/.style args={#1,#2,#3,#4,#5,#6,#7}{
		code = {
			% #1: node 1
			% #2: node 2
			% #3: node 1 guide line distance
			% #4: node 2 guide line distance
			% #5: label distance
			% #6: label
			% #7: label position (ex: right, left, above, below)
			\draw[very thin, gray] ($(#1)!#3!-90:(#2)$) -- ($(#1)!#5+3pt!-90:(#2)$); % lateral line perpendicular to nodes 1 and 2
			\draw[very thin, gray] ($(#2)!#4!90:(#1)$) -- ($(#2)!#5+3pt!90:(#1)$); % lateral line perpendicular to nodes 1 and 2
			\draw[very thin, arrows=<->, gray] ($(#1)!#5!-90:(#2)$) -- ($(#2)!#5!90:(#1)$) node[midway, #7] {#6}; % dimension line parallel to nodes 1 and 2
		}
	},
	pics/dimAngular/.style args={#1,#2,#3,#4,#5,#6,#7,#8}{
		code = {
			% #1: node 1 (this node will define the distance the arc from the central node)
			% #2: node 2
			% #3: node centre
			% #4: distance before arrow ends
			% #5: node 1 guide line distance
			% #6: label
			% #7: label position (ex: right, left, above, below)
			% Create lines for computing arc centre
			\def\argone{#4}\def\argtwo{0}
			\ifx\argone\argtwo % if the arrow offset is zero the rotation centre remains the same
				\coordinate (centre) at (#3);
			\else % if there is offset we find the:
				\draw[name path=AB,draw=none] ($(#2)!#4!90:(#3)$) -- ($(#3)!#4!-90:(#2)$); % line parallel to the link formed by the nodes 1 and 2 with offset #4
				\draw[name path=CD,draw=none] (#1) -- (#3); % reference line for measure the angle
				\coordinate [name intersections={of={AB and CD}}]; % intersection between the offset line and the reference line
				\coordinate (centre) at (intersection-1); % set the rotation centre as the found intersection
			\fi
			\draw [very thin, gray, #8] let
				\p0 = ($(#1)-(centre)$), % difference vector between reference and centre
				\p1 = ($($(#2)!#4!90:(#3)$)-(centre)$) % difference vector reference between the offset line and the centre
				in (#1) arc [start angle={atan2(\y0,\x0)}, end angle={atan2(\y1,\x1)}, radius=({veclen(\y0,\x0)})] node[midway, #7] {#6}; % angular dimension line
			\draw[very thin, gray] ($(#3)!#5!0:(#1)$) -- ($(#1)!-3pt!0:(#3)$); % Draw guide line
		}
	},
    pics/link/.style args={#1,#2,#3}{
		code = {
			% #1: node 1
			% #2: node 2
			% #3: characteristic dimension of drawing
            \draw[thick, fill=white] ($(#1)!1.5*#3!-90:(#2)$) -- ($(#2)!1.5*#3!90:(#1)$) -- ($(#2)!1.5*#3!-90:(#1)$) -- ($(#1)!1.5*#3!90:(#2)$) -- cycle; % lines parallel to nodes 1 and 2
            \fill[white, draw=none] (#2) circle circle (2.5*#3); % outer round part
            \draw[thick] (#2) circle circle (2.5*#3); % outer round part
            \fill[gray, draw=none] (#2) circle circle (#3);
            \draw[thick] (#2) circle circle (#3);
		}
	},
	pics/centreofmass/.style args={#1,#2}{
		code = {
			% #1: radius of symbol
			\draw[fill=black, thin] (#2) -- ++(#1,0) arc [radius=#1, start angle=0,end angle=90] -- ++(0,-2*#1) arc [radius=#1, start angle=270, end angle=180] -- cycle;
			\draw[thin] (#2) circle circle (#1);
		}
	},
	pics/lastlink/.style args={#1,#2,#3}{
		code = {
			% #1: node 1
			% #2: node 2
			% #3: characteristic dimension of drawing
            \coordinate (new2) at ($(#2)!2.0*#3!0:(#1)$);
			\draw[thick] ($(#1)!1.5*#3!-90:(new2)$) -- ($(new2)!1.5*#3!90:(#1)$);
			\draw[thick] ($(#1)!1.5*#3!90:(new2)$) -- ($(new2)!1.5*#3!-90:(#1)$);
			% End-effentor
			\draw[ultra thick,rounded corners=#3] ($($(new2)!2*#3!180:(#1)$)!3*#3!-90:(#1)$) -- ($(new2)!3*#3!-90:(#1)$) -- (new2) -- ($(new2)!3*#3!90:(#1)$) -- ($($(new2)!2*#3!180:(#1)$)!3*#3!90:(#1)$);
		}
	},
    pics/foot/.style args={#1,#2,#3}{
		code = {
			% #1: node 1
			% #2: node 2
			% #3: characteristic dimension of drawing
            \draw[fill=white, very thick, rounded corners=#3] ($(#1)!5.0*#3!-90:(#2)$) -- ($(#2)!1.5*#3!90:(#1)$) -- ($(#2)!1.5*#3!-90:(#1)$) -- ($(#1)!5.0*#3!90:(#2)$) -- cycle;
			%\draw[thick]  -- ;
			% End-effentor
			%\draw[ultra thick,rounded corners=#3] ($($(#2)!2*#3!180:(#1)$)!3*#3!-90:(#1)$) -- ($(#2)!3*#3!-90:(#1)$) -- (#2) -- ($(#2)!3*#3!90:(#1)$) -- ($($(#2)!2*#3!180:(#1)$)!3*#3!90:(#1)$);
            \fill[white, draw=none] (#2) circle circle (2.5*#3); % outer round part
            \draw[thick] (#2) circle circle (2.5*#3); % outer round part
            \fill[gray, draw=none] (#2) circle circle (#3);
            \draw[thick] (#2) circle circle (#3);
		}
	},
    pics/lastlinkcontact/.style args={#1,#2,#3}{
		code = {
			% #1: node 1
			% #2: node 2
			% #3: characteristic dimension of drawing
            % coordinates
            \coordinate (bottomLeft) at ($(#1)!1.5*#3!-90:(#2)$);
            \coordinate (bottomRight) at ($(#2)!1.5*#3!90:(#1)$);
            \coordinate (topLeft) at ($(#1)!1.5*#3!90:(#2)$);
            \coordinate (topRight) at ($(#2)!1.5*#3!-90:(#1)$);
            \coordinate (bottomRight2) at ($(bottomRight)!5.0*#3!0:(bottomLeft)$);
            \coordinate (topRight2) at ($(topRight)!5.0*#3!0:(topLeft)$);
            \coordinate (bottomRight3) at ($(bottomRight)!2.5*#3!0:(bottomLeft)$);
            \coordinate (topRight3) at ($(topRight)!2.5*#3!0:(topLeft)$);
            \pgfmathanglebetweenpoints{\pgfpointanchor{#1}{center}}{\pgfpointanchor{#2}{center}}
            \edef\angle{\pgfmathresult}
            % drawing
            \draw[thick, fill=white] (topLeft) -- (bottomLeft) -- ($(bottomRight2)!0.7!(bottomRight3)$) -- ($(topRight2)!0.7!(topRight3)$) -- cycle;
			% End-effentor
            \draw[thick, fill=gray] (topRight2) arc (\angle+90:\angle-90:1.5*#3) {[rounded corners=1.5pt] -- (bottomRight3)} arc (\angle+180:\angle+90:1.5*#3) coordinate (prov) {[rounded corners=1.5pt] arc (\angle+270:\angle+180:1.5*#3)} -- cycle;
            \draw[thick] (prov) -- (#2);
            \fill[black] (#2) circle (0.5pt);
		}
	},
    pics/limb/.style args={#1,#2,#3,#4}{
		code = {
			% #1: point coordinate 1
			% #2: point coordinate 2
			% #3: thickness of ellipse: 0 - line; 1 - circle
			% #4: scale: 1 is no scale
            % compute myangle and mylength:
            \getLengthAndAngle{#1}{#2}
            % compute central coordinate:
            \coordinate (centre) at ($(#1)!0.5!(#2)$);
            % draw limb
            \draw[rotate=\myangle,shift={(centre)}, very thick, fill=white] (0,0) ellipse ({#4*0.5*\mylength} and {#4*#3*0.5*\mylength});
		}
	},
    pics/square/.style args={#1,#2,#3,#4}{%
        code = {%
            % #1: square centre coordinate
            % #2: inclination angle
            % #3: square size length
			% #4: arrow color
            % ------------------------------
            % draw limb
            \shadedraw[shading angle=#2, cm={cos(-#2) ,-sin(-#2) ,sin(-#2) ,cos(-#2) ,(#1)}, thick, rounded corners=0.5pt] ($(#1)+(0.5*#3,0.5*#3)$) rectangle ($(#1)+(-0.5*#3,-0.5*#3)$);
            \draw[-stealth, #4, thick, cm={cos(-#2) ,-sin(-#2) ,sin(-#2) ,cos(-#2) ,(#1)}] (#1) --++ (0.45*#3,0);
            \fill[black] (#1) circle (1.0pt);
        }
    },
    pics/lastlinklinecontact/.style args={#1,#2,#3}{%
        code = {%
            % #1: node 1
            % #2: node 2
            % #3: characteristic dimension of drawing
            % ------------------------------
            % computing angle between nodes
            \pgfmathanglebetweenpoints{\pgfpointanchor{#1}{center}}{\pgfpointanchor{#2}{center}}
            \edef\angle{\pgfmathresult}
            % coordinates
            \coordinate (bottomRight) at ($(#2)!1.5*#3!90:(#1)$);
            \coordinate (bottomLeft) at ($(#1)!1.5*#3!-90:(#2)$);
            \coordinate (topRight) at ($(#2)!1.5*#3!-90:(#1)$);
            \coordinate (topLeft) at ($(#1)!1.5*#3!90:(#2)$);
            \coordinate (bottomMiddle) at ($(bottomRight)!2.0*#3!0:(bottomLeft)$);
            \coordinate (topMiddle) at ($(topRight)!2.0*#3!0:(topLeft)$);
            \coordinate (bottomBase) at ($(bottomRight)!2.0*#3!180:(topRight)$);
            \coordinate (topBase) at ($(topRight)!2.0*#3!180:(bottomRight)$);
            % drawing
            \draw[thick, fill=white] (bottomLeft) -- (bottomMiddle) -- (topMiddle) -- (topLeft) -- cycle;
            % End-effentor
            \draw[thick, fill=gray, rounded corners=0.5pt] (bottomMiddle)  to [out=\angle-90,in=\angle+135] (bottomBase) -- (topBase) to [out=\angle-135,in=\angle+90] (topMiddle) -- cycle;
        }
    },
    pics/linkwithbreak/.style args={#1,#2,#3}{%
        code = {%
            % #1: node 1
            % #2: node 2
            % #3: characteristic dimension of drawing
            \draw[thick, fill=white] ($(#1)!1.5*#3!-90:(#2)$) -- ($(#2)!1.5*#3!90:(#1)$) -- ($(#2)!1.5*#3!-90:(#1)$) -- ($(#1)!1.5*#3!90:(#2)$); % lines parallel to nodes 1 and 2
            \draw[very thin,line cap=round] ($(#1)!1.5*#3!-90:(#2)$) to [out=-135,in=45, thin] ($(#1)!1.5*#3!90:(#2)$);
            \fill[white, draw=none] (#2) circle circle (2.5*#3); % outer round part
            \draw[thick] (#2) circle circle (2.5*#3); % outer round part
            \fill[gray, draw=none] (#2) circle circle (#3);
            \draw[thick] (#2) circle circle (#3);
        }
    },
    pics/conveyercircle/.style args={#1,#2,#3}{%
        code = {%
            % #1: conveyercircle centre coordinate
            % #2: inclination angle
            % #3: conveyercircle size length
            % ------------------------------
            % draw circle
            \draw[black,fill=black] (#1) circle (0.8*#3);
            \draw[fill=gray!10!white] (#1) circle (0.4*#3);
            \draw[fill=gray!10!white, cm={cos(-#2) ,-sin(-#2) ,sin(-#2) ,cos(-#2) ,(#1)}] ($(#1)+(0.6*#3,0)$) circle (1pt);
            \draw[fill=gray!10!white, cm={cos(-#2) ,-sin(-#2) ,sin(-#2) ,cos(-#2) ,(#1)}] ($(#1)+(0,0.6*#3)$) circle (1pt);
            \draw[fill=gray!10!white, cm={cos(-#2) ,-sin(-#2) ,sin(-#2) ,cos(-#2) ,(#1)}] ($(#1)+(-0.6*#3,0)$) circle (1pt);
            \draw[fill=gray!10!white, cm={cos(-#2) ,-sin(-#2) ,sin(-#2) ,cos(-#2) ,(#1)}] ($(#1)+(0,-0.6*#3)$) circle (1pt);
        }
    },
}